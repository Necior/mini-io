\documentclass[a4paper]{article}

%% Language and font encodings
\usepackage[english]{babel}
\usepackage[utf8x]{inputenc}
\usepackage[T1]{fontenc}

%% Sets page size and margins
\usepackage[a4paper,top=3cm,bottom=2cm,left=3cm,right=3cm,marginparwidth=1.75cm]{geometry}

%% Useful packages
\usepackage{amsmath}
\usepackage{graphicx}
\usepackage[colorinlistoftodos]{todonotes}
\usepackage[colorlinks=true, allcolors=blue]{hyperref}

\title{The Project game}
\author{
Marcin Godniak \\
Tomasz Chudzi\\}

\begin{document}
\maketitle

\pagebreak
\section{Opis The Project Game}



\section{Zasady gry}

\subsection{Możliwe akcje}
\subsection{Reguły}

\section{Aktorzy}

\section{Możliwości konfiguracyjne}

\section{Uruchamianie aplikacji}

\section{opis jak gra będzie uruchamiana, co będzie wynikiem, gdzie konfiguracja itp}

\section{wymagania niefunkcjonalne - FURPS (m.in. ile graczy będzie mogła obsłużyć, co jeśli jeden z graczy się odłączy w trakcie itp)}

\section{ opis architektury}

\section{ protokół komunikacyjny (podłączanie, kończenie, żądanie ruchu, komunkacja itp)}

\subsection{diagramy sekwencji}
\subsection{struktura wiadomości (XML, JSON)}
\subsection{opis słowny}

\section{opis sytuacji wyjątkowych}
\subsection{utrata łączności z graczem}
\subsection{utrata łącznośći z CS}
\subsection{utrata łączności z GM}
  




\end{document}